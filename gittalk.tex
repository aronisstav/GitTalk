\documentclass[]{beamer}

\usepackage[cm-default]{fontspec}
\usepackage{xunicode}
\usepackage{xltxtra}
\setmainfont[Mapping=tex-text]{DejaVu Sans}

\usepackage{graphicx}
\usepackage{amssymb}

\usetheme{Pittsburgh}
\useinnertheme{rounded}
\usefonttheme{serif}
\usecolortheme{beaver}

\title{Εισαγωγή στο Git}
\author{Σταύρος Αρώνης}
\date{8 Ιουνίου 2011}
\institute{Κοινότητα Ελεύθερου Λογισμικού ΕΜΠ}

\setbeamertemplate{navigation symbols}{}
\setbeamertemplate{footline}{
  \leavevmode
  \hbox{
    \begin{beamercolorbox}[wd=.5\paperwidth,ht=2.5ex,dp=1.125ex,right]
      {author in head/foot}
      \usebeamerfont{title in head/foot}
      \insertshortauthor
      \hspace{.3cm}
    \end{beamercolorbox}%
    
    \begin{beamercolorbox}[wd=.40\paperwidth,ht=2.5ex,dp=1.125ex,left]
      {title in head/foot}
      \usebeamerfont{author in head/foot}\
      \hspace{.3cm}
      \insertshorttitle
    \end{beamercolorbox}
    
    \begin{beamercolorbox}[wd=.10\paperwidth,ht=2.5ex,dp=1.125ex,center]
      {title in head/foot}
      \usebeamerfont{author in head/foot}
      \insertframenumber/\inserttotalframenumber
    \end{beamercolorbox}
  }
  \vskip0pt
}

%% \AtBeginSection[]
%% {
%%   \begin{frame}<beamer>
%%     \frametitle{Επισκόπηση}
%%     \tableofcontents[currentsection, hideothersubsections]
%%   \end{frame}
%% }

\begin{document}

\begin{frame}
  \titlepage
  \begin{center}
    \includegraphics[height=2cm]{Fosstux.png}
    \hspace{1cm}
    \includegraphics[height=2cm]{git-logo.png}
  \end{center}
\end{frame}

\section*{Επισκόπηση}
\begin{frame}
  \tableofcontents[hidesubsections]
\end{frame}

\section{Revision Control Systems}

\subsection{Ορισμός}

\begin{frame}
  \frametitle{Revision Control Systems}
  \begin{center}
    \includegraphics[height=7cm]{hotfix-branches1.png}
  \end{center}
\end{frame}

\begin{frame}
  \frametitle{Revision Control Systems}
  \begin{itemize}
    \item Χρησιμοποιούνται σε projects με πολύ κώδικα ή/και πολλούς
      collaborators.
    \item Καταγράφουν συστηματικά:
      \begin{itemize}
        \item ποιός
        \item πως
        \item πότε και
        \item γιατί
      \end{itemize}
      αλλάζει τον κώδικα.
    \item Μπορούν να χρησιμοποιηθούν αποτελεσματικά και σε εργασίες που δεν
      περιέχουν κώδικα αλλά γενικότερο text περιεχόμενο (\TeX{} documents,
      etc)
  \end{itemize}
\end{frame}

\begin{frame}
  \frametitle{Λεξιλόγιο}
  \begin{itemize}
  \item \textbf{Repository:} Το φυσικό σημείο στο οποίο βρίσκονται αποθηκευμένα
    τα αρχεία μαζί με όλες τις πληροφορίες του RCS (local ή remote).
  \item \textbf{Commit:} Ένα σύνολο λογικά συσχετιζόμενων αλλαγών στα
    παρακολουθούμενα αρχεία (edit, add, delete) μαζί με πληροφορίες για το ποιός
    τις έκανε, πότε και γιατί.
  \item \textbf{Branch:} Όταν υπάρχει ανάγκη για απόσχιση από το ``κεντρικό
    σώμα'' του κώδικα (π.χ. για ένα experimental feature ή bug fix), ένα branch
    κρατάει τα νέα commits χωρίς να επηρεάζει το σώμα του κώδικα.
  \item \textbf{Tag:} Δείκτης σε μια συγκεκριμένη έκδοση του κώδικα.
  \item \textbf{Merge:} Η διαδικασία ενσωμάτωσης ενός branch σε ένα άλλο.
  \end{itemize}
\end{frame}

\section{Distributed RCS}

\begin{frame}
  \frametitle{Distributed RCS}
  \begin{center}
    \includegraphics[height=7cm]{centr-decentr.png}
  \end{center}
\end{frame}

\begin{frame}
  \frametitle{Central RCS}
  \begin{itemize}
    \item \textbf{Βασική ιδέα:} Η ``επίσημη έκδοση'' του κώδικα βρίσκεται σε ένα
      κεντρικό repository.
    \item Κάθε commit αποθηκεύεται κεντρικά και μόνο εκεί.
      \begin{center}
        \includegraphics[height=2cm]{social_network.jpg} \hspace{0.5em}
      \end{center}\pause
      \begin{itemize}
      \item Δεν μπορεί ο καθένας να κάνει commit.
        \includegraphics[width=1em]{Lock1.png} \pause
      \item Πρέπει να υπάρχει σύνδεση στο δίκτυο για πρόσβαση στα commits (log,
        history, diff, blame).
      \end{itemize}
  \end{itemize}
\end{frame}

\begin{frame}
  \frametitle{Distributed RCS}
  \begin{itemize}
    \item \textbf{Δεν υπάρχει ``επίσημη έκδοση'' του κώδικα! Κάθε developer έχει
      δικό του repository!}\pause
    \item Οι collaborators έχουν \textbf{όλο} το project τοπικά:
      \begin{itemize}
        \item Κώδικας
        \item Commits
      \end{itemize}\pause
    \item Κάνουν commit τοπικά, χωρίς να επηρεάζουν κανέναν άλλο.\pause
    \item ``Πηδάνε'' από το ένα topic στο άλλο δημιουργώντας branches
      τοπικά.\pause
    \item Ανταλλάσσουν commits και συγχρονίζουν τα repositories τους.
  \end{itemize}
\end{frame}

\section{Git}

\begin{frame}
  \frametitle{Git}
  \framesubtitle{\url{http://git-scm.com/}}
  \begin{itemize}
  \item Projects using Git
    \begin{itemize}
    \item Git
    \item Linux Kernel
    \item Perl
    \item Eclipse
    \item Gnome
    \item KDE
    \item Qt
    \item Ruby on Rails
    \item Android
    \item PostgreSQL
    \item Debian
    \item X.org
    \end{itemize}
  \end{itemize}
\end{frame}

\begin{frame}
  \frametitle{Git}
  \begin{itemize}
    \item Κάποια από τα χαρακτηριστικά που κάνουν το git να ξεχωρίζει:
      \begin{itemize}
        \item Εύκολο branching \pause
        \item \textbf{Εύκολο merging} \pause
        \item Έλεγχος εγκυρότητας των δεδομένων (SHA1)\pause
        \item Απελπιστικά γρήγορο! \pause
        \item Awesome UI \pause
      \end{itemize}
    \item Και όπως και άλλα πιθανώς DRCS: \pause
      \begin{itemize}
        \item Network of trust security model \pause
        \item Full access to logs and diffs
      \end{itemize}
  \end{itemize}
\end{frame}

\begin{frame}
  \frametitle{Github}
  \framesubtitle{\url{https://github.com/}}
  Χρησιμεύει σαν ένα κεντρικό sync point:
  \begin{center}
    \includegraphics[height=7cm]{centr-decentr.png}
  \end{center}
\end{frame}

\section{Using Git}

\begin{frame}
  \frametitle{Using Git}
  \tableofcontents[currentsection, hideothersubsections]
\end{frame}

\begin{frame}
  \frametitle{Using git}
  \framesubtitle{Προκαταρκτικά...}
  \begin{itemize}
    \item Στήσιμο pseudo-central server ή Λογαριασμός στο github
    \item SSH keys για πιστοποίηση των collaborators
    \item Στήσιμο repository
  \end{itemize}
\end{frame}

\subsection{New project}

\begin{frame}
  \frametitle{Using git}
  \framesubtitle{New project}
  \begin{itemize}
    \item git init
    \item git add
    \item git commit (-m, --amend)
    \item git diff(tool)
    \item git remote add
    \item git push
  \end{itemize}
\end{frame}

\subsection{Existing project}

\begin{frame}
  \frametitle{Using git}
  \framesubtitle{Existing project}
  \begin{itemize}
    \item GitHub fork
    \item git clone
    \item git add
    \item git commit (-m, --amend)
    \item git push
  \end{itemize}
\end{frame}

\subsection{Collaboration}

\begin{frame}
  \frametitle{Using git}
  \framesubtitle{Collaboration}
  \begin{itemize}
    \item git remote update
    \item git merge
    \item git pull
    \item git mergetool
  \end{itemize}
\end{frame}

\subsection{Experimentation}

\begin{frame}
  \frametitle{Using git}
  \framesubtitle{Experimentation}
  \begin{itemize}
    \item git branch
    \item git checkout
    \item git push remote new\_branch
    \item make
    \item git merge (again)
  \end{itemize}
\end{frame}

\subsection{History}

\begin{frame}
  \frametitle{Using git}
  \framesubtitle{History}
  \begin{itemize}
    \item git log
    \item git tree
    \item git show
    \item gitk
  \end{itemize}
\end{frame}

\subsection{Advanced}

\begin{frame}
  \frametitle{Advanced}
  \framesubtitle{History}
  \begin{itemize}
    \item Let's find the bug: git bisect
    \item Polish a patch: git rebase
    \item ACHIEVEMENTS!
  \end{itemize}
\end{frame}

\section*{References}

\begin{frame}
  \frametitle{References}
  \begin{itemize}
    \item Github tutorials
    \item Git tutorials
    \item Git man pages
    \item Torvalds talk @ Google
    \item Full management model:
      \url{http://nvie.com/posts/a-successful-git-branching-model/}
    \item Git achievements
  \end{itemize}
\end{frame}

\begin{frame}
  \frametitle{Git tutorial}
    \begin{center}
      Thank you!
    \end{center}
\end{frame}

\end{document}

% xelatex -interaction=nonstopmode gittalk.tex; evince gittalk.pdf
